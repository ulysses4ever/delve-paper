\section{Introduction}

The current approach to building data analysis applications is
dominated by languages like Python, R, and, to some extent, Julia.
All of these languages can be considered as general-purpose; e.g. it is easy
to build a web server in Julia, and less so in R, but also possible.
This generality calls for factorizing implementation of various tasks into
library code, as well as gives vast freedom as to which language
features to employ when building applications. This clearly has downsides.

General-purpose languages take the libraries-based approach to tackle
data processing (e.g. Pandas in Python, \dplyr in R, DataTables in Julia).
Such libraries pose the challenge of learning the APIs with many possible
caveats and subtle differences between each other.
Another common issue with this approach is tracking changes in APIs, e.g.
functions receive new parameters, become deprecated, etc.

Imperative features of the said languages, especially unchecked mutation
of the global state, may hinder understanding of a script and the grounds
of its validity. This often leads to subtle errors, and bugs in data science
scripts are a topic of many recent studies
(consider, for example, a recent study of bugs in COVID-19 related software~\cite{bugscovid}).

In the area of data analysis,
both of the issues (the API-based approach and imperative features)
can be avoided by employment of a declarative
domain-specific language targeted at this particular kind of tasks.
As a bonus, a significantly special-purposed DSL shall
allow efficient compilation to a robust application~--- something that
has been an issue with languages like Python and R for decades.
On the the down side, a declarative interface would lack support
for inherently imperative tasks faced by data analysts  (e.g. file system manipulation).
This can be resolved by embedding the DSL into a general-purpose
language.

In this paper, we showcase a commercial implementation of Datalog
embedded in Julia, called Delve, that checks all the above boxes: it is declarative,
has virtually zero amount of API vocabulary needed to apply it,
and it can resort to the efficient JIT compiler of Julia for the tasks
not amenable for declarative processing. Delve's backend is build on solid
foundations of database systems.

We consider a data analysis application we call Truck Factor.
Two implementations are suggested, using Delve and R. We compare them
along two axes. First, linguistics: how much of code is required,
how much of ``external'' vocabulary (the one not concerned with the particular
application) a programmer need to learn to build such implementation.
Second, performance and scalability: we consider a spectrum from
toy loads to the ones not fitting in computer RAM.

\section{Truck Factor}

For a case study, we aimed at a just-above-toy application having to
do with data analysis. To this end, we picked a task of computing the
truck factor (also known as bus factor) of a software repository
roughly following the methodology of~\cite{tf} including the
degree-of-authorship formula~\cite{doa} as the basis of the metric.
We did not aim to reproduce results~\cite{tf}
especially because their dataset of 133 GitHub repositories is insufficient
for our purposes: to make any conclusion performance-wise, we needed
thousands of repositores. In this section we discuss data format we work with,
the algorithm to compute the truck factor, and representative code snippets
from both implementations, in R and in Delve.

\subsection{Dataset}

We use open source projects hosted on GitHub and using Git as their
version control history. The history, or metadata, is what analysis
uses as the input with the caveat that it works with CSV-formatted
tables instead of binary Git-objects. In order to convert the latter
to the former we use an auxiliary tool called
GhGrabber\footnote{\url{https://github.com/PRL-PRG/ghgrabber}}.
We do not analyze the source code itself.

\subsection{Algorithm}

The purpose of the algorithm is to establish the number of a project
(i.e. software repository) contributors who ``authors'' at least half of
the source files. The degree-of-authorship of a file $f$ (represented
by a filepath $f_p$)
for a developer $d$ (mapped to a GitHub identity $m_d$)
is determined using the following formula~\cite{tf}:
\begin{multline}
\DOA(m_d, f_p) =
  3.293 + \\
  1.098\times \FA(m_d, f_p) +
  0.164\times \DL(m_d, f_p) - \\
  0.321\times \ln(1 + \AC(m_d, f_p)).
\end{multline}
Here, first authorship (FA) is $1$ if $m_d$ originally created $f$~---
otherwise it is $0$; number of deliveries ($\DL$): the number of changes in $f$
made by $m_d$; finally, number of acceptances ($\AC$): the number of
contributions in $f$ made by any developer, except $m_d$. The coefficients
are picked up empirically~\cite{doa}.

The high-level structure of the algorithm is as follows: (i) for every project
source file and every developer touched that file compute the three
parameters mentioned above ($\FA$, $\DL$, $\AC$) and the final metric ($\DOA$);
(ii) for every file, pick its developer with the maximum $\DOA$ as the file's
author; (iii) sort developers in decreasing order by the number of files
they author and count how many of the ``top'' developers author at least
half of files.

\subsection{R implementation}

As the basis for the R implementation of the algorithm we use a popular high-level
pipeline-oriented DSL \dplyr\footnote{%
\url{https://dplyr.tidyverse.org/}%
}. It provides a set of combinators (called ``verbs'' in the documentation)
familiar from SQL and some other relational-programming languages. Among those
combinators:
\texttt{select},
\texttt{filter},
\texttt{arrange},
\texttt{summarise}.

Here is an example of \dplyr in action for computing the number of contributions
(measured as the number of commits that touched the file, according to Git) by individual
developers~--- this is useful for computing the $\AC$ metric mentioned above:
\begin{verbatim}
  main %>%
    group_by(uid,author) %>%
    summarize(n=n()) ->
    contributions
\end{verbatim}
The \m{main} table consists of records of contributions that various developers
submitted to files in various projects.
We generate unique global identifiers (\m{uid}'s) for every file in the
dataset of projects in advance. Probably, the most confusing part of \dplyr's
syntax here is the \m{n()} function: it counts the number of elements in the
group, and the result is written down in the column conventionally named \m{n}
too. The result of this pipeline is stored in a new table called \m{contributions}.


\subsection{Delve implementation}

Delve is a Datalog implemented as a Julia package. The frontend is standalone,
so a Delve program has to be written in a separate file (or in a string literal in a Julia
source file) and passed to the \m{query\_delve} function in the Julia space.
The backend, on the other hand, is tightly integrated with the Julia runtime, so
the computed result comes out as a collection of plain Julia objects. Therefore,
the outputs can be post-processed in a Julia program
(e.g. printed to the console or to a CSV file).

The example from the previous subsection (computing the number of contributions)
is expressed in Delve as follows:
\begin{verbatim}
  def contributions = uid author contribution :
    main(uid,_,author,_,_,_) and
    (count[hash: main(uid,hash,author,_,_,_)])(contribution)
\end{verbatim}
Here, \m{def} is a keyword signaling that we are about to define a new relation
(\m{contributions} in this case). The relation will have three components
listed between \m{=} and \m{:}, namely \m{uid}, \m{author}, and \m{contribution}.
After the colon we provide a formula for the new relation. The \m{uid} and
\m{author} components should participate in the \m{main} relation. For every
such values we compute the value of \m{contribution} using the auxiliary
\m{count} combinator. The input to \m{count} is a relation (in this case it is
defined inline): it has one column \m{hash} and every row is the hash
identifier of a commit authored by the \m{author} and touching the file
identified by \m{uid}. Th output of \m{count} is again a relation but having just one row
with one field indicating the number of rows in the input relation.

\section{Evaluation}

When trying to compare different implementations of the same algorithm, there
are several possible view points. Two important ones are linguistic parameters
and performance.

\subsection{Linguistics}

Table~\ref{tab1-lingua} compares linguistics of both implementations modulo data
loading. The number of lines is similar, which is important because line count
is one of the primary measures of code size.

What stands out in Table~\ref{tab1-lingua} is the word count for Delve: the
corresponding implementation is far wordier than the one in R. But if we look
at the number of unique words, it is a parity again~---just as with the line count.

There is an important difference in the distribution of unique words though.
In particular, among the top five most-frequent words in the Delve
implementation three are the column names; the other two are \m{def} and \m{and}.
In contrast, for the R implementation, among the five top words: two are the
names specific to the \dplyr vocabulary, (\m{group\_by}, \m{summarize}), another two
are application-specific (\m{doa}, \m{uid}), and the last one is somewhere in
between: it is \m{n}~--- the name of a \dplyr function, as well as the column name
that we use to store results of calls to this function.

\begin{table}
\setlength{\tabcolsep}{12pt}
\caption{Linguistic characteristics of both implementations.}\label{tab1-lingua}
\centering
\begin{tabular}{lcc}
\toprule
              &  R      & Delve \\
\midrule
Lines         & 107     & 116\\
Words         & 277     & 450\\
Unique Words  &  77     & 79 \\
\bottomrule
\end{tabular}
\end{table}

% TODO: add table for top 5 unique words
Another interesting metric of vocabulary diversity is distribution of
words frequencies.
Occurrences of five most-frequent words in Delve implementation account
for 42\% of all words.
For R, this metric is 21\%~---exactly two times less than for Delve! This
shows that a Delve programmer most of the time thinks about the data representation,
whereas an R programmer needs to care about various parts of the API simultaneously
with application-specific identifiers.

\subsection{Performance}

Delve is a Datalog implementation with the focus on performance and large-scale
applications, whereas it was never a strong suit for R. We performed several
experiments that mostly proved this observation with the caveat that for tiny
datasets R managed to outperform Delve a bit. This can be explained by
overhead of a full-fledged database back-end employed by Delve.

In particular, on a set of 20 projects (20 Mb) R took 2.2 seconds for computations,
while Delve required 4.7 seconds. In contrast, on a set of 1000 projects (254 Mb)
R runs for 38 seconds, while Delve finishes in 14 seconds.

Delve authors advertise its ability to handle larger-than-RAM dataset
transparently. R has several packages that attempt to provide the same ability
but this is much less explored space for this language. We haven't studied
performance of our implementations at this scale, although it could be an
interesting future work.

\section{Conclusion}

Over-talkative API of \dplyr and uncontrolled mutation available in R bit us
several times during this work. For instance, certain \dplyr functions changed
interface slightly even during not-so-long period of time of this work, so we
had to adjust. The ability to mutate tables served as a source of subtle bugs:
an updated instance of a table was used sometimes when the original one was expected.

In comparison, working with Delve, although requiring a good
amount of typing, is free from these issue and much easier to read after some
break~--- most likely, because it speaks in application-specific terms as
opposed to API-specific. With industrial-strength compilers such as Delve, we
conclude that Datalog should be seriously considered as a tool for tasks in
data analysis.


