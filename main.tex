% This is samplepaper.tex, a sample chapter demonstrating the
% LLNCS macro package for Springer Computer Science proceedings;
% Version 2.20 of 2017/10/04
%
\documentclass[runningheads]{llncs}
%
\usepackage{graphicx}
% Used for displaying a sample figure. If possible, figure files should
% be included in EPS format.

% If you use the hyperref package, please uncomment the following line
% to display URLs in blue roman font according to Springer's eBook style:
%\renewcommand\UrlFont{\color{blue}\rmfamily}

\usepackage{booktabs}

\usepackage{amsmath}
\DeclareMathOperator{\DOA}{DOA}
\DeclareMathOperator{\FA}{FA}
\DeclareMathOperator{\DL}{DL}
\DeclareMathOperator{\AC}{AC}

\usepackage{xspace}
\newcommand{\dplyr}{\texttt{dplyr}\xspace}
\newcommand{\m}{\texttt}

\begin{document}
%
\title{Exploring Datalog for Modern Data Analysis}
\subtitle{(Application paper)}
%
%\titlerunning{Abbreviated paper title}
% If the paper title is too long for the running head, you can set
% an abbreviated paper title here
%
\author{%
Artem Pelenitsyn%\inst{1}%\orcidID{0000-1111-2222-3333}
%\and
%Second Author\inst{1}%\orcidID{1111-2222-3333-4444}
}
%
\authorrunning{A. Pelenitsyn}
% First names are abbreviated in the running head.
% If there are more than two authors, 'et al.' is used.
%
\institute{Northeastern University}
%
\maketitle              % typeset the header of the contribution
%
\begin{abstract}
The current approach to building data analysis applications is dominated by
imperative languages armed with robust libraries for typical data processing
tasks. Both imperative features and APIs pose well-known challenges that can
be avoided if a declarative domain-specific language for data manipulation is
employed. In this paper, we showcase a commercial implementation of Datalog
embedded in Julia, called Delve, that has a number of plummy features: it is
declarative, has virtually zero amount of API vocabulary required to apply it,
and can resort to the efficient JIT compiler of Julia for the tasks not
amenable for declarative processing. We discuss two implementations of a data
processing application, using Delve and R, and compare them along two axes:
linguistics and performance.


%The abstract should briefly summarize the contents of the paper in
%150--250 words.

\keywords{Datalog  \and Julia \and R \and embedded languages.}
\end{abstract}
%
%
%

\section{Introduction}


\section{Truck Factor}

\subsection{Algorithm}

\subsection{R implementation}

\subsection{Delve implementation}

\section {Discussion}

\section{Conclusion}


%
% ---- Bibliography ----
%
% BibTeX users should specify bibliography style 'splncs04'.
% References will then be sorted and formatted in the correct style.
%
\bibliographystyle{splncs04}
\bibliography{refs.bib}
%
\end{document}
